% Texroot: skript.tex
\usepackage{xpatch}

% Colors
\definecolor{dunkelgrau}{cmyk}{0,0,0, 0.75} %{0.25, 0.25, 0.30, 0.75}
\definecolor{goetheblau}{cmyk}{1.00, 0.20, 0.00, 0.40}
\definecolor{hellgrau}  {cmyk}{0.04, 0.04, 0.05, 0.02}
\definecolor{sandgrau}  {cmyk}{0.12, 0.09, 0.13, 0.00}
\definecolor{purple}    {cmyk}{0.08, 1.00, 0.30, 0.36}
\definecolor{emorot}    {cmyk}{0.04, 1.00, 0.80, 0.07}
\definecolor{senfgelb}  {cmyk}{0.01, 0.25, 1.00, 0.05}
\definecolor{gruen}     {cmyk}{0.62, 0.40, 0.87, 0.09}
\definecolor{magenta}   {cmyk}{0.08, 0.86, 0.12, 0.12}
\definecolor{orange}    {cmyk}{0.00, 0.70, 1.00, 0.04}
\definecolor{sonnengelb}{cmyk}{0.00, 0.12, 0.95, 0.00}
\definecolor{hellgruen} {cmyk}{0.04, 0.17, 0.81, 0.07}
\definecolor{lichtblau} {cmyk}{0.80, 0.00, 0.06, 0.04}
\colorlet{tablegrey}{sandgrau!50}
\colorlet{sidecolor}{dunkelgrau}
\colorlet{labelcolor}{goetheblau}

\colorlet{fillColor}{goetheblau!30}
\colorlet{fillColorAlt}{sandgrau}
\colorlet{fillColorStrong}{emorot!40}
\colorlet{frontColor}{black}
\colorlet{frontColorAlt}{goetheblau}
\colorlet{frontColorStrong}{emorot}

\colorlet{color1}{goetheblau}
\colorlet{color2}{emorot}
\colorlet{color3}{orange}
\colorlet{color4}{senfgelb}
\colorlet{color5}{lichtblau}


% Header/Footer	
\usepackage{fancyhdr}
\fancyhf{}
\def\headerFormat{\sffamily\color{dunkelgrau}}
\def\defaultPageStyle{
	\pagestyle{fancy}
	\fancyhead[RO]{\headerFormat\nouppercase{\rightmark}}
	\fancyfoot[LE,RO]{\headerFormat\thepage}
}
\defaultPageStyle
\xpretocmd\headrule{\color{dunkelgrau}}{}{\PatchFailed}

\renewcommand{\sectionmark}[1]{\markright{#1}}
\renewcommand{\subsectionmark}[1]{}
\setcounter{secnumdepth}{3}

% Side nodes
\usepackage{marginnote}
\usepackage{sidenotes}
\usepackage[font={small}, labelfont={sf,color=labelcolor},textfont={sf}]{caption}
\renewcommand*{\marginfont}{\small\color{sidecolor}\sffamily\slshape}
\newcommand{\aside}[2][0pt]{\marginnote{#2}[#1]}
\sidecaptionvpos{figure}{t}
\sidecaptionvpos{table}{t}
\renewcommand*\sidecaptionrelwidth{0.75}
\captionsetup[table]{format=plain, font={color=sidecolor,small}}
\DeclareCaptionStyle{marginfigure}{format=plain, labelfont={sf,color=labelcolor,small}, textfont={sl,sf,small}}
\DeclareCaptionStyle{sidecaption}{format=plain, font={color=sidecolor, small}, labelfont={sf,color=labelcolor}, textfont={sl,sf}}
\DeclareCaptionStyle{sidecapfmt}{format=plain, font={color=sidecolor, small}, labelfont={sf,color=labelcolor}, textfont={sl,sf}}

\usepackage{refcount}% http://ctan.org/pkg/refcount
\newcounter{oddpagecheck}
\makeatletter
\newcommand{\ifoddpageelse}{% \ifoddpage{<odd>}{<even>}
	\stepcounter{oddpagecheck}% For unique labels
	\label{opc-\theoddpagecheck}% Place label
	\ifodd\getpagerefnumber{opc-\theoddpagecheck}
		\expandafter\@firstoftwo% Page is odd
	\else
		\expandafter\@secondoftwo% Page is not odd (even)
	\fi}
\makeatother

\newenvironment{widefigure}[1][t]{%
	\begin{figure}[#1]
		\ifoddpageelse{}{\hspace{-\margintotalwidth}}
		\begin{minipage}{\textwidth+\margintotalwidth}
			}{
		\end{minipage}
	\end{figure}
}


\newenvironment{widetable}[1][t]{%
	\begin{table}[#1]
		\ifoddpageelse{}{\hspace{-\margintotalwidth}}
		\begin{minipage}{\textwidth+\margintotalwidth}
			}{
		\end{minipage}
	\end{table}
}

\newenvironment{widebox}[1][\margintotalwidth]{%
	\ifoddpageelse{}{\hspace{\dimexpr-#1\relax}}%
	\begin{minipage}{\textwidth+#1}%
		}{%
	\end{minipage}%
}

\newenvironment{widemath}[1][\margintotalwidth]{%
	\ifoddpageelse{}{\hspace{\dimexpr-#1\relax}}%
	\begin{minipage}{\textwidth+#1}%
		}{%
	\end{minipage}%
}

\newcommand{\raggedinner}{\ifoddpageelse{\raggedright}{\raggedleft}}
\newcommand{\raggedouter}{\ifoddpageelse{\raggedleft}{\raggedright}}


% Pseudo code
\usepackage[ruled,vlined,linesnumbered]{algorithm2e}
\DontPrintSemicolon
\SetAlFnt{\small}

\newcommand{\mycommentsty}[1]{\textcolor{goetheblau}{\sffamily #1}}
\SetCommentSty{mycommentsty}

\newcommand{\mydatasty}[1]{{\sffamily \textsc{\textcolor{gruen}{#1}}}}
\SetDataSty{mydatasty}

\newcommand{\mykwfunctiondatasty}[1]{\textsc{#1}}
\SetFuncSty{mykwfunctiondatasty}


\usepackage{newfloat}

\DeclareFloatingEnvironment[fileext=algflt,placement={!ht},name=Algorithm]{algoflt}
\urlstyle{sf}

% Theorems et al.
\RequirePackage{amsthm}
\usepackage{thmtools}
\usepackage{thm-restate}
\usepackage{mdframed}


\declaretheoremstyle[
	headfont={\sffamily\color{labelcolor}},
	headpunct=\textcolor{labelcolor}{.},
	postheadspace=1em,
	qed=\textcolor{labelcolor}{$\blacktriangleleft$}
]{endmarker}

\declaretheorem[style=endmarker, parent=chapter]{theorem}
\declaretheorem[style=endmarker, numbered=no, name=Theorem]{theorem*}

\declaretheorem[style=endmarker, numberlike=theorem]{definition}
\declaretheorem[style=endmarker, numbered=no, name=Definition]{definition*}
\declaretheorem[style=endmarker, numberlike=theorem]{corollary}
\declaretheorem[style=endmarker, numbered=no, name=Korollar]{corollary*}
\declaretheorem[style=endmarker, numberlike=theorem]{lemma}
\declaretheorem[style=endmarker, numbered=no, name=Lemma]{lemma*}
\declaretheorem[style=endmarker, numberlike=theorem]{proposition}
\declaretheorem[style=endmarker, numbered=no, name=Proposition]{proposition*}
\declaretheorem[style=endmarker, numberlike=theorem]{conjecture}
\declaretheorem[style=endmarker, numbered=no, name=Conjecture]{conjecture*}
\declaretheorem[style=endmarker, numberlike=theorem]{example}
\declaretheorem[style=endmarker, numbered=no, name=Example]{example*}
\declaretheorem[style=endmarker, numbered=no, name=Note]{note*}
\declaretheorem[style=endmarker, numberlike=theorem, name=Beobachtung]{observation}
\declaretheorem[style=endmarker, numbered=no, name=Beobachtung]{observation*}
\declaretheorem[style=endmarker, numberlike=theorem]{remark}
\declaretheorem[style=endmarker, numbered=no, name=Remark]{remark*}
\declaretheorem[style=endmarker, numberlike=theorem, name=Aufgabe]{exercise}
\declaretheorem[style=endmarker, numbered=no, name=Aufgabe]{exercise*}

\declaretheorem[style=endmarker, numberlike=theorem,
	%	mdframed={backgroundcolor=black!10, linecolor=labelcolor, topline=false,rightline=false},
	qed={},
	name={Case Study}]{casestudy}

\declaretheorem[style=endmarker, numberlike=theorem, mdframed={
			backgroundcolor=black!10,
			linecolor=labelcolor,
			topline=false,rightline=false},
	qed={},
	name={Example}]{introexample}

\expandafter\let\expandafter\oldproof\csname\string\proof\endcsname
\let\oldendproof\endproof
\renewenvironment{proof}[1][\proofname]{%
	\renewcommand\qedsymbol{\textcolor{goetheblau}{$\square$}}
	\oldproof[\normalfont\textcolor{goetheblau}{\sffamily #1}]%
}{\oldendproof}



\RequirePackage{aliascnt}
\usepackage{titlesec}
\usepackage{mathrsfs}
\usepackage{multirow}


\newcommand{\fmtLabel}[1]{\textcolor{darkgray}{\sffamily\bfseries #1}}

\makeatletter
\renewcommand\section{\@startsection {section}{1}{\z@}%
  {-3.5ex \@plus -1ex \@minus -.2ex}%
  {2.3ex \@plus.2ex}%
  {\sffamily\large\bfseries\raggedright}}
\renewcommand\subsection{\@startsection{subsection}{2}{\z@}%
	{-3.25ex\@plus -1ex \@minus -.2ex}%
	{1.5ex \@plus .2ex}%
	{\sffamily\large\bfseries\raggedright}}
\renewcommand\subsubsection{\@startsection{subsubsection}{3}{\z@}%
	{-3.25ex\@plus -1ex \@minus -.2ex}%
	{1.5ex \@plus .2ex}%
	{\sffamily\large\bfseries\raggedright}}
\renewcommand\paragraph{\@startsection{paragraph}{4}{\z@}%
	{-3.25ex \@plus-1ex \@minus-.2ex}%
	{1.5ex \@plus .2ex}%
	{\sffamily\large\bfseries\raggedright}}
\renewcommand\subparagraph{\@startsection{subparagraph}{5}{\z@}%
	{3.25ex \@plus1ex \@minus .2ex}%
	{-1em}%
	{\sffamily\normalsize\bfseries}}
\makeatother



\usepackage{listings}% http://ctan.org/pkg/listings
\lstset{
	basicstyle=\ttfamily,
	mathescape
}


\makeatletter
\newenvironment{relatedbib}{%
	\begingroup%
	\makeatletter%
	\let\@bibitem\saved@bibitem%
	\nobibliography*%
	\begin{itemize}%
		\def\sc{} % PREVENT AUTHORS IN SC
		\def\bibentryitem##1{\item[\cite{##1}] \bibentry{##1} .}%
		      \def\bibentryitemaside##1##2{\item[\cite{##1}]\aside{##2}\bibentry{##1} .}%
		      }{%
	\end{itemize}%
	\endgroup%
}

\colorlet{chapterDigitColor}{dunkelgrau!50}

\newcommand{\chapterTitlePage}[3]{%
	\titleformat{\chapter}[display]
	{}
	{} % label \thechapter
	{0pt} % sep
	{} % before
	[] % after		

	\clearpage{\thispagestyle{empty}\cleardoublepage}
	\chapter[#2]{
	  \vspace{-3cm}
	  \begin{tikzpicture}
		  \node[anchor=south east, chapterDigitColor, inner sep=0] (cpt-mark) {\begin{minipage}{1.25\textwidth}\raggedleft\scalebox{5}{\Huge \thechapter}\end{minipage}};
		  \node[font={\huge}, anchor=south west, at=(cpt-mark.south west), inner sep=0] (cpt-title) {\begin{minipage}{\textwidth}\sffamily#3\end{minipage}};
	  \end{tikzpicture}
	 }
	\label{cpt:#1}
	\thispagestyle{empty}
}


\newenvironment{paperfrontmatter}[1]{%
	\def\@chapter@title{}
	\def\@chapter@runninTitle{}
	\def\@chapter@abstract{}
	\def\@chapter@remarks{}
	\let\@chapter@authors\relax
	\def\@chapter@basedon{}
	\let\@chapter@contribution\relax

	\newcommand{\cTitle}[1]{\gdef\@chapter@title{##1}}
	\newcommand{\cAuthors}[1]{\gdef\@chapter@authors{##1}}
	\newcommand{\cRunningTitle}[1]{\gdef\@chapter@runningTitle{##1}}
	\newcommand{\cAbstract}[1]{\long\gdef\@chapter@abstract{##1}}
	\newcommand{\cRemarks}[1]{\long\gdef\@chapter@remarks{##1}}
	\newcommand{\cBasedOn}[1]{\gdef\@chapter@basedon{##1}}
	\newcommand{\cContribution}[1]{\gdef\@chapter@contribution{##1}}

	\newcommand{\paperChapterTitlePage}{%
		\chapterTitlePage{#1}{\@chapter@runningTitle}{\@chapter@title}
		\ifthenelse{\equal{\@chapter@authors}{}}{%
			\vspace{-5em}%
		}{%
			\vspace{-5em}%
			{
				\linespread{1.3}
				\sffamily joint work with \@chapter@authors\par
			}
			\vspace{1em}%
		}%
		\vfill
		\hfill
		\begin{minipage}{0.96\textwidth}%
			\sffamily%
			\@chapter@abstract
		\end{minipage}%
		%		
		\vfill\vfill
		\noindent%		
		\@chapter@remarks
		\begin{relatedbib}
			\foreach \x in \@chapter@basedon {\bibentryitem{\x}}
		\end{relatedbib}
		\ifx\@chapter@contribution\relax\else
			\vfill
			\subsubsection*{My contribution}
			\@chapter@contribution
		\fi
		%	
		\vfill\vfill
		\vfill\vfill
		%	
		\clearpage%
	}
}{%
	\paperChapterTitlePage%
}

\newcommand{\startpaper}[1]{%
	\fancyhead[LE]{\headerFormat\@chapter@runningTitle}%
}

\makeatother

\colorlet{DarkGreen}{green}
\colorlet{DarkBlue}{blue}
\colorlet{DarkRed}{red}

\tikzset{snake it/.style={decorate, decoration=snake}}
\tikzset{onif/.code args={<#1>#2}{\ifthenelse{#1}{\pgfkeysalso{#2}}{}}}
\tikzset{
conflict/.style={DarkRed, fill=red!10},
state1/.style={DarkGreen, fill=green!10},
state2/.style={DarkBlue, fill=blue!5},
state3/.style={Maroon, fill=Yellow!10},
state4/.style={Red, fill=Thistle!30},
state5/.style={Black, fill=black!5},
state?/.style={conflict}
}
\def\arrowHead{Latex[length=0.5em, width=0.5em]}

\newcommand{\labelfmt}[1]{\textcolor{labelcolor}{\textsl{#1}}}

\newcommand{\seeref}[1]{\leftpointright~\cref{#1}}
\newcommand{\qremark}[1]{\labelfmt{[#1]}\xspace}
\newcommand{\qmremark}[1]{\text{\textcolor{labelcolor}{\ensuremath{#1}}}}
\renewcommand{\path}[1]{\IfSubStr{#1}{_}{\StrSubstitute{#1}{_}{\_}}{#1}}


\newcommand{\reffmt}[1]{\textcolor{labelcolor}{#1}}
\newcommand{\labrefint}[4]{\reffmt{#1:} #2 \reffmt{\seeref{#3} #4}\\}
\newcommand{\labref}[3][]{\labrefint{#2}{\\}{#3}{#1}}
\newcommand{\labrefff}[2]{\labrefint{#1}{\\}{#2}{ff.}}
\newcommand{\sidelabref}[2]{\aside{\labref{#1}{#2}}}

\newenvironment{subapx}{
	\addcontentsline{toc}{section}{Appendix}
	\begin{subappendices}
		\settocdepth{chapter}
		}{
		\resettocdepth
	\end{subappendices}
}

\newcommand{\floatmarginfigure}[1]{
	\begin{figure}[h]%
		\aside{\ \par \vspace{-0.5em} #1}%
		\vspace{-1.5em}
	\end{figure}
}